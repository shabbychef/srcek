\documentclass[10pt]{letter}
\textheight=9in\textwidth=6.5in\topmargin=0in\oddsidemargin=0in

\usepackage[newitem,newenum,increaseonly]{paralist}
\usepackage[commands]{sepmath}

\address{Steven E. Pav\\
Technology Development\\
Nellcor\\
4280 Hacienda Drive\\
Pleasanton, CA, 94588 USA\\
Steven.Pav\@@TycoHealthcare.com}
\name{Steven E. Pav}

\location{Tech Dev}
\telephone{925 463 4253}
\signature{Steven E. Pav}

\date{\today}

\begin{document}

\begin{letter}{Professor C.H. Spiegelman\\%FOLDUP
Statistics Department\\
Texas A \& M University\\
College Station, TX 77843, USA}%UNFOLD
\opening{Professor Spiegelman;}

Please consider the attached manuscript, ``SRCEK: A Continuous Embedding of the
Channel Selection Problem for WPLS Modeling,'' for publication in the Journal
of Chemometrics and Intelligent Laboratory Systems.  This paper explores a
continuous embedding of the problem of selecting a subset of available
channels, or wavelengths, for affine modelling of a response by weighted PLS.
This paper includes a formulation of the WPLS algorithm which also allows
computation of the Jacobian of the regression vector with respect to predictor
weightings.  This allows one to minimize the quality of cross validation with
respect to continuous weightings.  Following a numerical optimization, the
embedding is reversed \emph{e.g.} by use of an information criterion.

In addition to those potential reviewers listed in the attached file, I might
also suggest that Michele Forina review this paper, as I believe he is
eminently qualified for the task.  However, I have had some brief communications 
with Dr. Forina regarding this paper (he graciously supplied the data sets), so
I wish to avoid any impressions of impropriety by suggesting him as a reviewer.

I look forward to hearing from you soon.  If you have any questions, do not
fail to contact me.  

\closing{Yours truly,}
%\encl{AMS cover sheet\\curriculum vita\\list of publications\\statement of research}
%\encl{AMS cover sheet\\curriculum vita\\list of publications\\
%statements of research and teaching}
\encl{List of Potential Reviewers,\\Manuscript}

\end{letter}
%FOLDUP
%My current research focuses on the problem of dividing arbitrary, closed,
%affinely bounded domains in $\reals{2}$ and $\reals{3}$ into simplicial
%complices suitable for application of the finite element method.  To insure
%convergence of the method with respect to shrinking element size, it is desireable for
%the output simplices to be ``nicely shaped.'' One measure of this quality is
%that simplices should have a low circumradius-to-inradius ratio.  In
%particular, I am studying the Delaunay Refinement method, a well-known
%technique for producing high-quality Delaunay ``meshes'' of planar domains;
%the method produces meshes where small simplices are near small features of the
%input, and moreover has an optimality result:  it produces meshes with no more
%than a constant (albeit an astronomical one) as many Steiner points as
%\emph{any} other technique which solves the same meshing problem.  In practice
%the technique works rather well, not approaching the theoretically possible
%poor performance of the optimality theorem.
%
%The main results of my thesis are as follows:
%\begin{compactitem}
%\item The Delaunay Refinement Algorithm, in \reals{2}, can be applied to a
%wider class of input than previously analyzed. In fact any well-formed input
%can be used, subject to a preprocessing step.  The algorithm is optimal in all
%cases, but the proven optimality constant becomes arbitrarily large for input
%with small angles.  The algorithm does suffer some loss of output quality for
%input with small angles.  
%\item A locally adaptive variant of the algorithm, which is not too different
%from what has already been widely implemented, can guarantee the presence of
%high quality output simplices far from small input angles.  Thus the poor
%quality guarantee of the regular algorithm can be locally isolated.  
%\item By using a more ``aggressive'' preprocessing step, a better output
%quality can be guaranteed by the adaptive algorithm.  It can be guaranteed that
%no angles smaller than about $26.4\dgry$ will be present in the output mesh, an
%improvement over the previously known bound of about $20.7\dgry.$  (In
%\reals{2}, the quality measure is equivalent to smallest angle.  It appears
%that angle guarantees greater than $30\dgry$ are not possible.)
%The tradeoff for the output guarantee is a small constant possible increase in
%the number of Steiner Points.  
%
%\end{compactitem}
%
%Additionally, I have shown that the Delaunay Refinement method can be
%generalized to arbitrary dimensions.  Some unfortunate facts about this work
%are that assumptions on the input are far too restrictive, there isn't really
%an optimality result, and the output quality guarantee is terrible, \ie a
%weaker measure than inradius-circumradius is controlled.
%
%In ongoing and future research, I hope to
%\begin{compactitem}
%\item Concoct a robust alteration of the Delaunay Refinement method for
%        meshing in \reals{3} (or even \reals{n}), that can accept arbitrary input. A
%				locally adaptive version would be acceptable.  This currently looks
%				promising.
%\item Close the gap from the current $26.4\dgry$ output
%        angle bound to the theoretical maximum value of $30\dgry.$
%\item Study the problem of ``sliver removal'' in \reals{3}, \ie 
%guaranteeing output mesh quality via the circumradius-inradius measure
%rather than the currently known measure, which may leave behind ``flat''
%tetrahedra known as slivers.
%\end{compactitem}
%
%
%UNFOLD
\end{document}
%lost and found:
%\begin{letter}{Professor Hans-Peter Seidel\\%FOLDUP
%Max-Planck-Institut f�r Informatik\\
%Stuhlsatzenhausweg 85\\
%66123 Saarbr\"{u}cken\\
%Germany}
%\opening{Dear Professor Seidel;}%UNFOLD
%\begin{letter}{Kurt Mehlhorn\\%FOLDUP
%Max-Planck-Institut f\"ur Informatik\\
%Im Stadtwald\\
%D-66123 Saarbr\"ucken\\
%Germany}
%\begin{letter}{
%Prof. Dr. E. Zeidler\\
%Managing Director\\
%MPI for Mathematics in the Sciences\\
%Inselstr. 22\\ 
%D-04103 Leipzig\\
%Germany}
%\opening{Dear Dr. Zeidler;}%UNFOLD
%\begin{letter}{Professor Ersin Yurtseve\\%FOLDUP
%Dean of the College of Arts and Sciences\\
%Ko\c{c} University\\
%Rumelifeneri Yolu, 34450 Sariyer-Istanbul\\
%Turkey}
%\opening{Dear Dr. Yurtseve;}%UNFOLD
%\begin{letter}{Professor Hans Petter Langtangen\\%FOLDUP
%Department of Scientific Computing\\
%Simula Research Laboratory\\
%P.O. Box 134\\
%1325 Lysaker\\
%Norway}%UNFOLD
%\begin{letter}{Hiring Committee\\%FOLDUP
%Department of Mathematics\\
%San Francisco State University\\
%San Francisco, CA, 94132}
%\opening{To whom it may concern}%UNFOLD
%\begin{letter}{Attn: Job Code NSAM002\\%FOLDUP
%David B. Ashley\\
%Executive Vice Chancellor and Provost\\
%University of California, Merced\\
%PO Box 2039\\
%Merced, CA 95344-0039.}
%\opening{Dear Vice Chancellor Ashley;}%UNFOLD
%\begin{letter}{Hiring Committee\\%FOLDUP
%Department of Mathematics\\
%San Francisco State University\\
%San Francisco, CA, 94132}
%\opening{To whom it may concern;}%UNFOLD
%\begin{letter}{Search Committee\\%FOLDUP
%Department of Mathematics\\
%Stanford University\\
%Stanford, CA 94305}
%\opening{To whom it may concern;}%UNFOLD
%\begin{letter}{Szeg\"{o} Search Committee\\%FOLDUP
%Department of Mathematics\\
%Stanford University\\
%Stanford, CA 94305}
%\opening{To whom it may concern;}%UNFOLD
%\begin{letter}{attn: \#05-06 MCS-MATHEMATICS-TT\\%FOLDUP
%Department of Math/CS\\
%California State University\\
%Hayward, CA 94542-3092}
%\opening{To whom it may concern;}%UNFOLD
%My work is primarily in the area of computational geometry, in particular the %FOLDUP
%problem of dividing arbitrary, closed, affinely bounded domains in $\reals{2}$
%and $\reals{3}$ into simplicial complices suitable for application of the
%finite element method.  The work that comprises my thesis analyzes the
%termination and optimality of the Delaunay Refinement method of Ruppert;  I
%have discovered a slight variant of the algorithm which terminates with an
%optimality guarantee independent of small angles in the input domain.
%Moreover, I have discovered that Ruppert's original strategy for dealing with
%midpoint-midpoint interactions works, producing optimal meshes, and can
%guarantee better output meshes than previously thought.
%
%For future research, I hope to address the following issues:
%\begin{compactitem}
%\item Currently there is no known strategy for applying the Delaunay Refinement
%	method, in \reals{3}, to domains with arbitrarily small dihedral angles.  I
%	would like to generalize the analysis I have done for the two dimensional
%	case to develop such a strategy.
%\item Another failure of the quality meshing in three (and higher) dimensions
%	is the problem of ``slivers,'' that is, simplices with small
%	inradius-circumradius ratio, but with a large shortest edge-circumradius ratio.
%	To devise some means of dealing with such simplices would be of great
%	advantage for the application of the finite element method.
%\item In my thesis, I show that the Delaunay Refinement Algorithm can guarantee
%	angles in the output are no smaller than $26.4\dgry.$  I would like to
%	somehow answer the open question: ``Is there a technique for generating
%	well-graded meshes with a $30\dgry$ output guarantee?''  (Naturally, the real
%	output guarantee would have to somehow accommodate small input angles.)
%\end{compactitem}
%
%%I have asked the following people to send letters of reference separately:
%
%I encourage you to contact the following references:
%\begin{compactitem}
%\item[] Noel J. Walkington, Department of Mathematics, Carnegie Mellon
%University.  noelw\@@andrew.cmu.edu
%\item[] Gary L. Miller, School of Computer Science, Carnegie Mellon
%University.  glmiller\@@cs.cmu.edu
%\item[] Shlomo Ta'asan, Department of Mathematics, Carnegie Mellon
%University.  shlomo\@@andrew.cmu.edu
%\end{compactitem}%UNFOLD
%FOLDUP
%\begin{compactitem}
%\item Noel J. Walkington, Department of Mathematics, Carnegie Mellon
%University.  noelw\@@andrew.cmu.edu
%\item Jonathan Shewchuk, Computer Science Division, University of California
%at Berkeley.  jrs\@@cs.berkeley.edu
%\item Tom Bohman, Department of Mathematics, Carnegie Mellon
%University.  tbohman\@@moser.math.cmu.edu
%\item Gary L. Miller, School of Computer Science, Carnegie Mellon
%University.  glmiller\@@cs.cmu.edu
%\item Shlomo Ta'asan, Department of Mathematics, Carnegie Mellon
%University.  shlomo\@@andrew.cmu.edu
%\end{compactitem}%UNFOLD
